\documentclass[a4paper,12pt]{report}
\usepackage[utf8]{inputenc}
\usepackage[italian]{babel}
\usepackage[italian]{cleveref}
\usepackage{anyfontsize}
\usepackage{fancyhdr}
\usepackage{changepage}
\usepackage{geometry}

\usepackage{booktabs} % For prettier tables
\usepackage{graphicx}
\usepackage{array}
\usepackage[table]{xcolor}
\usepackage[section]{placeins}
\usepackage{float}
\usepackage{tabto}
\usepackage{comment}
\usepackage{listings}
\usepackage{color}
\usepackage{enumitem}

\usepackage{etoolbox}
\makeatletter
\patchcmd{\chapter}{\if@openright\cleardoublepage\else\clearpage\fi}{}{}{}
\makeatother


\title{\textbf{Relazione progetto} \break Traccia 3 - ``progetto''}
\author{Montali Giacomo - 0000925283 \\ Savini Edoardo - 0000XXXX}
\date{Giugno 2021}

\begin{document}
	\maketitle
	{\fontsize{14}{20}\selectfont
		\chapter{Introduzione}
			Il progetto implementa un’architettura client-server per il supporto di un gioco multiplayer testuale in rete locale.
			
		\chapter{Descrizione}
			Partendo dall'idea di base del chat game abbiamo deciso di sviluppare un gioco stile `inserire stile`, nel quale i giocatori si sfideranno con delle domande, che forniranno loro dei punti in base alla risposta data, il tutto accompagnato da una chat che permetterà loro di comunicare.
			Inoltre nella schermata sarà presente un riquadro contenente la classifica in tempo reale. 
			Le domande alle quali i giocatori dovranno rispondere avranno 4 risposte, ma solo una di queste sarà corretta. Per complicare il tutto, è stato inserito un timer, allo scadere del quale si passerà alla prossima domanda. \\
			Prima dell'avvio della partita però, sarà richiesto al client di selezionare il proprio username e di connettersi al server che hosterà il gioco.
			Il gioco ha un limite di XXX giocatori.
			
			\section{Scelte effettuate}
				\begin{description}[font=$\bullet$~\normalfont\scshape\color{red!90!black}]
					\item Utilizzo di una comunicazione client-server basata su oggetti JSON,
					per semplificarne la gestione.
					
					\item Gestione dell'interazione dell'utente con le risposte in maniera facile ed intuitiva grazie ai 4 pulsanti inseriti.
					
					\item 
					
					\item
				\end{description}
		
		\chapter{Dettagli implementativi}
			minchiate
			\section{Strutture utilizzate}
				\begin{description}[font=$\bullet$~\normalfont\scshape\color{red!90!black}]
					\item 
											
					\item 
											
					\item
				\end{description}
				
			\section{Thread utilizzati}
				\begin{description}[font=$\bullet$~\normalfont\scshape\color{red!90!black}]
					\item Thread Receiver, situato nei client, gestisce i messaggi ricevuti dal server per poi eseguire le operazioni necessarie per l'aggiornamento del gioco.
								
					\item Thread Timer, che permette al programma di impostare un limite di tempo per ogni domanda.
								
					\item
				\end{description}
				
			\section{Librerie utilizzate}	
				\begin{description}[font=$\bullet$~\normalfont\scshape\color{red!90!black}]
					\item tkinter
					\item thread
					\item json
					\item time
					\item socket, AF\_INET, SOCK\_STREAM
				\end{description}
	}
\end{document}
